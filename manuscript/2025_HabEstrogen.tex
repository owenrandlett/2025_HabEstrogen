
%%% PREAMBLE 
\documentclass[9pt,lineno]{RandlettLab_elife}
\nolinenumbers
\usepackage{lipsum} % Required to insert dummy text
\usepackage{listings}
\usepackage[version=4]{mhchem}
\usepackage{siunitx}
\usepackage{gensymb}
\DeclareSIUnit\Molar{M}
\usepackage{cancel}
\usepackage{xcolor}
\usepackage{textgreek}
\definecolor{codegreen}{rgb}{0,0.6,0}
\definecolor{codegray}{rgb}{0.5,0.5,0.5}
\definecolor{codepurple}{rgb}{0.58,0,0.82}
\definecolor{backcolour}{rgb}{0.96,0.96,0.96}

\lstdefinestyle{mystyle}{
    backgroundcolor=\color{backcolour},   
    commentstyle=\color{codegreen},
    keywordstyle=\color{magenta},
    numberstyle=\tiny\color{codegray},
    stringstyle=\color{codepurple},
    breakatwhitespace=false,         
    breaklines=true,                 
    captionpos=b,                    
    keepspaces=true,                 
    numbers=left,                    
    numbersep=5pt,                  
    showspaces=false,                
    showstringspaces=false,
    showtabs=false,                  
    tabsize=2
}

\lstset{style=mystyle}

%%%%%%%%%%%%%%%%%%%%%%%%%%%%%%%%%%%%%%%%%%%%%%%%%%%%%%%%%%%%
%%% ARTICLE SETUP
%%%%%%%%%%%%%%%%%%%%%%%%%%%%%%%%%%%%%%%%%%%%%%%%%%%%%%%%%%%%
\title{Estradiol increases visual habituation learning independently of the canonical estrogen receptors}

\author[ !,1,2] 
{Andrew Hsiao}

\author[ !,1] 
{Isabelle Darvaux-Hubert}

\author[ 1,3] 
{Dominique Hicks}

\author[ 1,2] 
{Emilie Joux}

\author[ 1,2]
{Sarah De Freitas}

\author[ 1,2]
{Emeline Dracos}

\author[ 1,2]
{Jeanne Litze}

\author[ *,1] 
{Dominique Baas}

\author[ *,@,1] 
{Owen Randlett}


\affil[1]{
Laboratoire MeLiS, Université Claude Bernard Lyon 1 - CNRS UMR5284 - Inserm U1314, Institut NeuroMyoGène, Faculté de Médecine et de Pharmacie, 8 avenue Rockefeller, 69008 Lyon, France
}

\affil[2]{
International Master in Life Sciences, Université Claude Bernard Lyon 1, France
}

\affil[3]{
Master of Biology Program, École normale supérieure de Lyon, France
}

\affil[!]{equal contribution}

\affil[*]{equal contribution}


\affil[@]{correspondence: \href{mailto:owen.randlett@univ-lyon1.fr}{owen.randlett@univ-lyon1.fr}}

%%%%%%%%%%%%%%%%%%%%%%%%%%%%%%%%%%%%%%%%%%%%%%%%%%%%%%%%%%%%
%%% ARTICLE START
%%%%%%%%%%%%%%%%%%%%%%%%%%%%%%%%%%%%%%%%%%%%%%%%%%%%%%%%%%%%

\begin{document}

\maketitle
\begin{abstract}

Habituating to the constant stimuli in the environment is a critical adaptive learning process conserved across species. 
We use the larval zebrafish visual response to sudden darkness as a model for studying habituation learning, where zebrafish reduce their responses to repeated stimulations. 
In this paradigm, treatment with Estradiol strongly increases learning rate, resulting in reduced responses. 
In an attempt to identify the receptor(s) mediating these effects we used established mutant lines with expected null alleles for the known estrogen receptors (\emph{esr1}, \emph{esr2a}, \emph{esr2b}, \emph{gper1}). 
Our experiments failed to identify a receptor required for the effects of Estradiol on habituation learning. 
Surprisingly, nuclear-receptor mutants showed increased habituation relative to sibling controls when treated with estradiol, indicating that activation of these receptors has paradoxical inhibitory effects on habituation learning. 
These experiments confirm that Estradiol is a potent modulator of learning in the vertebrate brain, but suggest that these effects occur independently of the classical estrogen receptor-mediated signaling pathways, which may in fact act to inhibit learning performance in this paradigm.  

\end{abstract}

\section{Introduction}

A primary task of the brain is to learn from ongoing experiences and adjust behavior accordingly. 
This often involves sharpening attention and behavioural resources toward salient cues while tuning out irrelevant background stimuli. 
For instance, it may be critical to recognize the alarm calls of a nearby animal, whereas continually registering a steady hum from distant traffic is far less important. 
The capacity to reduce responses to repetitive, non-essential stimuli is known as habituation, a phenomenon widely considered one of the simplest forms of learning and memory \citep{Rankin2009-no}. 

We have been studying a paradigm for long-term habituation where larval zebrafish reduce their responsiveness to sudden pulses of whole-field darkness, or dark flashes (DFs) \citep{wolman_chemical_2011, Randlett2019-fj, Lamire2023-he}. 
In this analysis, we emphasize long-term habituation as a practical model for examining the fundamental processes that shape neural circuit plasticity.
We recently reported that multiple hormonal signaling pathways show strong modulation of habituation learning performance, including Melatonin, progesterone, and estrogen \citep{Lamire2023-he}. 
The ability of these signaling pathways to modulate learning is consistent with previous results in other systems and paradigms \citep{Nilsson2002-as, Naderi2020-ot, Dillon2013-rk, Rawashdeh2007-bw, Jilg2019-oy, El-Sherif2003-vt, Barros2015-jm}, and may be an important mechanism to shift learning and memory performance or strategies based on biological rhythms or external fluctuations like seasons, weather or the day/night cycle.

In this project we have focused on estrogens, 


Estrogen signaling in learning and memory, any data about receptors. 

In this project we aimed to identify the relevant estrogen receptor mediating the effects of estradiol analogs on habituation learning by undertaking a classical analysis of genetic knockout alleles. Not only did we fail to identify a mutant (or combination of mutants) that lead to estradiol instensitivity, but our results indicat that esrX receptors act to inhibit habituation learning -- effects which are usually masked by an unidentified estradiol-responsive pathway that promotes learning, resulting in estrogne receptor mutants that show increased sensitivity to estradiol in the habituation paradigm. 


%\newpage
\section{Materials and Methods}

\subsection{Animal Ethics Statement}

Adult zebrafish used to generate larvae were housed in accordance with PRCI facility approved by the animal welfare committee (comité d’éthique en expérimentation animale de la Région Rhône-Alpes: CECCAPP, Agreement \# C693870602). 
Behaviour experiments were performed at the 5dpf stage, and are thus not subject to ethical review, but these procedures do not harm the larvae. 

\subsection{Animals}

All experiments were performed on larval zebrafish at 5 days post fertilization (dpf), raised at a density of $\approx$1 larvae/mL of E3 media supplemented with 0.02\% HEPES pH 7.2. 
Larvae were raised in a 14:10h light/dark cycle at 28-29\degree{}C. 
Adult zebrafish were housed, cared for, and bred at the Lyon PRECI zebrafish facility. 
Mutant lines were obtained from D. Gorelick's lab, and were of the following alleles: 

\emph{esr1\textsuperscript{uab118}} is a CRISPR/Cas9-generated allele with a 4bp deletion (ZDB-ALT-180420-2), yielding a predicted null frameshift/stop mutation, confirmed by a lack of estradiol responsiveness in the heart as assayed by \emph{Tg(5xERE:GFP)\textsuperscript{c262}} expression \citep{Romano2017-ep}. 

\emph{esr2a\textsuperscript{uab134}} is a 2bp deletion (ZDB-ALT-180420-3), yielding a predicted null frameshift/stop mutation \citep{Romano2017-ep}

\emph{esr2b\textsuperscript{uab127}} is a 4bp deletion (ZDB-ALT-180420-4), yielding a predicted null frameshift/stop mutation, confirmed by a lack of estradiol responsiveness in the liver as assayed by \emph{Tg(5xERE:GFP)\textsuperscript{c262}} expression \citep{Romano2017-ep}. 

\emph{gper1\textsuperscript{uab102}} is a 133bp deletion (ZDB-ALT-180420-1), yielding a predicted null frameshift/stop mutation, confirmed by a lack of estradiol responsiveness in heart beating rate in maternal-zygotic mutants \citep{Romano2017-ep}.

\subsection{Genotyping}

Larvae were genotypes after behavioural analysis using.......

\subsection{Pharmacology}

\textbeta-Estradiol (Sigma E2758, here referred to as "estradiol") was dissolved in dimethyl sulfoxide (DMSO) and stored at -20\degree C. 
Larvae were treated with estradiol immediately before the behavioural assay by pipetting 10-30uL of 10x solution directly into the behavioural wells, always with a final concentration of 0.1\% DMSO in E3.

\section{Results and Discussion}

\subsection{Nuclear Estrogen receptors are not required for the effects of estradiol on habituation learning}

\subsection{Conclusion}

Here I described our system for tracking the tail of the larval zebrafish during microscopy. Many of the practical considerations of this setup may be specific to our application, and therefore may need modification for use in other experiments in other labs. However, I feel that the core and simple idea of using an IR-sensitive Raspberry Pi Camera, a simple Python script, coupled with IR LEDs and and IR filter, provides an approachable and flexible solution that may be widely useful for observing and tracking the behaviour of zebrafish (or perhaps other animals) while performing imaging experiments. This system's attributes may also make it an ideal tool for community engagement activities such as school outreach programs. It could serve as a platform for learning about microelectronics, behavioural analyses, machine vision, and hardware design and construction.

\section{Funding}

This work was supported by funding from the ATIP-Avenir program of the CNRS and Inserm, a Fondation Fyssen research grant, and the IDEX-Impulsion initiative of the University of Lyon.

\section{Data Availability}

Software and analysis code is available here:  \href{https://github.com/owenrandlett/pi_tailtrack/}{https://github.com/owenrandlett/pi\_tailtrack/}. Datasets are available here: \href{https://www.dropbox.com/sh/dbjq2dud1ws1o2v/AACLamthISys8sUD1a5oRcR1a?dl=0}{pi\_tailtrack datasets}.

\bibliography{References_HabEstrogen}

\end{document}
supplement